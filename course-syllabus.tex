\documentclass[10pt]{article}
\usepackage{hyperref}

\linespread{1.23}

\usepackage[a4paper,rmargin={0.5in},lmargin={0.5in},bmargin={0.5in},tmargin={0.5in}]{geometry}
\usepackage{color}
\begin{document}
\thispagestyle{empty}
\begin{center}
{\large\textsc {Syllabus}} \\
{\large MATH 1700 Section 101, Fall 2025} \\
TTh 11:00 - 12:15pm, Cudahy Hall 001 
\end{center}

\begin{tabular}{ll}
\hspace{-.25in}\textbf{Instructor:} Mehdi Maadooliat, Ph.D. & \textbf{TA:} Mobina Pourmoshir \\ 
e-mail: \href{mailto:mehdi.maadooliat@marquette.edu}{mehdi.maadooliat@marquette.edu} & \hspace{.25in} e-mail: \href{mailto:mobina.pourmoshir@marquette.edu}{mobina.pourmoshir@marquette.edu} \\
website: \href{http://www.mscs.mu.edu/~mehdi}{http://www.mscs.mu.edu/ $\widetilde{ }$mehdi} & \hspace{.25in} Office Hours:  MW 3:00 - 6:00pm \\
Office Hours: T (CU 351) \& Th (HelpDesk) 12:15 - 1:30pm & \hspace{1.1in} {TTh 2:15 - 3:15pm} \\
Office: CU 351 & \hspace{.25in} MTTh (HelpDesk) \& W (\href{https://tinyurl.com/StatHelpDesk}{MS Teams})
\end{tabular}
%%%%%%%%%%%%%%%%%%%%%%%%%%%%%%%%%%%%%%%%%
\begin{description}

\item[Textbook:] Elementary Statistics, 11th edition, by Johnson and Kuby, 2012.\\
\textbf{\textcolor{red}{(You need to acquire the WebAssign account, which includes the ebook.)}}
\vspace{-.1in}
\item[Calculators:] You will need some sort of scientific calculator for the course.
\item[Core of Common Studies Mathematical Reasoning Learning Outcomes:] You should be able to:\vspace{-.1in}
\begin{itemize}
\item[1.] Evaluate the effectiveness of the mathematical sciences in describing the world.\vspace{-.1in}
\item[2.] Analyze quantitative information symbolically, graphically, numerically, and verbally for the purpose \vspace{-.1in}
\item[] of solving problems or drawing conclusions.\vspace{-.1in}
\item[3.] Construct logical arguments in support of mathematical assertions. \vspace{-.1in}
\end{itemize}

\item[Learning Objectives:] \ \vspace{-.1in}
\begin{itemize}
\item[1.] Understand a few necessary concepts of probability\vspace{-.1in}
\item[2.] Understand the difference between descriptive statistics and inferential statistics.\vspace{-.1in}
\item[3.] Understand the estimation problem.\vspace{-.1in}
\item[4.] Understand the hypothesis problem.\vspace{-.1in}
\item[5.] Understand the difference between $3$ and $4$.\vspace{-.1in}
\item[6.] Calculate Linear Correlation and Line of Best Fit.
\end{itemize}

\item[Homework:] Online homework will be assigned and graded through WebAssign.\\
WebAssign page: \url{https://www.webassign.net/}\qquad\qquad\qquad\qquad WebAssign class key: {\color{blue}marquette 1594 2988}\\ 
Assignments are \textbf{typically} due Fridays at 11:50pm (check for the \textbf{exact} due dates in WebAssign). 

\item[Exams:] There will be two {\bf{70 minutes long}} on Sept. 25th, and Oct. 23rd, plus\\
\hspace*{5.2mm} a {\bf{120 minutes long}} final: Dec. 12th from 3:30 - 5:30pm. \vspace{-.1in}

\item \textbf{MAKE-UP POLICY:} If you have an ``unavoidable absence'' as defined in Arts and Sciences Undergraduate Bulletin, the percent of the missed Exam will be added to your Final Exam percentage. \textbf{Contact me if is University event absence.}

\item[Websites:] Lecture Notes on D2L : \url{http://d2l.mu.edu/}. WebAssign Assignments here: \url{https://www.webassign.net/}

\item[Grading:] Examinations  \quad 25\% each - 2 \qquad\qquad\qquad Final Exam  \quad 30\% \\
\hspace*{7.5mm} Attendance \qquad 5\% \qquad\qquad\qquad\qquad\qquad Homework \qquad 15\%\vspace{-.1in}
 
Everyone must be given the same opportunity to do well in this class. Individual exams WILL NOT be curved; however, I \textbf{MAY} use class participation to make adjustments at the end of the semester. The final grade is based on a scale no stricter than:\vspace{-.05in}

\hspace*{7.5mm} $93.5\leq x\leq 100$\hspace*{2.9mm}  = (A) \hspace*{10mm} $90\leq x<93.5$ = (A-) \hspace*{10mm} $86.5\leq x<90$ = (B+)\\
\hspace*{7.5mm} $83.5\leq x<86.5$\hspace*{2mm} = (B) \hspace*{10mm} $80\leq x<83.5$ = (B-) \hspace*{10mm} $76.5\leq x<80$ = (C+)\\
\hspace*{7.5mm} $73.5\leq x<76.5$\hspace*{2mm} = (C) \hspace*{10mm} $70\leq x<73.5$ = (C-) \hspace*{10mm} $66.5\leq x<70$ = (D+)\\
\hspace*{7.5mm} $60.0\leq x<66.5$\hspace*{2mm} = (D) \hspace*{10mm} below $60$ \hspace*{5mm}  \ = (F)\vspace{-.05in}

\item[Drop Date:] The last day to drop without a W is 09/02/2025 and with a W is 11/14/2025.

\item[WebAssign Support:] - 
\begin{enumerate}
\item[1.] Here’s a website which provides registration instructions and purchase option information. \url{https://www.cengage.com/coursepages/Marquette_WebAssign}
\item[2.] Here are video walkthrough instructions: \href{https://startstrong.cengage.com/webassign-not-integrated-ia-no/}{How to Register for WebAssign}: \url{https://startstrong.cengage.com/webassign-not-integrated-ia-no/}
\item[3.] Paul Valecce from Cengage offers Zoom Help Desk Hours for WebAssign registration and you can find the Zoom link and his office hours on the website \url{https://www.cengage.com/coursepages/Marquette_WebAssign}
\end{enumerate}

\item[Course Description:] Fundamental theory and methods of statistics without calculus. Descriptive statistics, elements of probability theory, estimation, tests of hypotheses, regression, correlation, introduction to computer methods of statistical tabulation and analysis (using Microsoft Excel). This course is recommended for students seeking a general introduction to statistical concepts and is not intended to be a final course in statistics for students who need a thorough working knowledge of statistical methods. Prerequisite: MATH 105 or equivalent. Equivalent is two years of college preparatory mathematics. This course may not be taken for credit by students who have received college credit for another probability or statistics course.

\item [Grading Style for Exams:] Your work is considered, not just a final answer. It is possible to have the wrong answer but receive (almost) full credit for a problem. If your work does not support your answer, it is unlikely that you will receive many points even if your answer agrees with the solution key.

\item [Grading Style for Homework:] The homework is designed to provide you with practice using the concepts taught in the course. Practice is essential to be prepared for the tests in this class. Online homework will be assigned and graded through WebAssign. These assignments will be grades with a $20\%$ adjustment. This means that if you score $80\%$ or higher for your online homework you will receive the full $15\%$ towards your overall score. If you score less than $80\%$, I will calculate your percentage as $\frac{\mathrm{Your\ percentage}}{80}\%$. In other words, if you end the semester with $70\%$ you will receive $\frac{70}{80}=87.5\%$ for your homework score.

\item [Attendance:] Attendance is required. The College of Arts and Sciences’ attendance policy will be observed. 

\item[Academic Support] You are encouraged to obtain help from me, the course instructor, the course TA in {\color{red} HelpDesk (25 hours per week)}. However, if you miss class, you are responsible for the missed material/information covered.\vspace{.1in}

\item [Tutoring:] \textbf{How can I be successful in this course?}

Join the free small group tutoring sessions. It is built-in study time with your classmates and led by a student who excelled in this course and wants to help you do the same.  Sessions begin on September 5th, but you can sign up for your weekly tutoring session as early as the first day of class!   Do it today.  Visit the website: \url{https://www.marquette.edu/tutoring/} for the sign-up link and more information.

\item[Academic Coaching:] \url{https://www.marquette.edu/student-educational-services/coaching.php}

\item[Deportment:] Norms for classroom conduct are based on respect for the instructor and fellow students. While in class, please silence your cell phones; do not text, update your status on social media sites, etc. Using cell phones, reading newspapers, sleeping, playing games, etc., distract fellow students.

\item[Honor Pledge and Honor Code:]  I recognize the importance of personal integrity in all aspects of life and work. I commit myself to truthfulness, honor, and responsibility, by which I earn the respect of others. I support the development of good character, and commit myself to uphold the highest standards of academic integrity as an important aspect of personal integrity. My commitment obliges me to conduct myself according to the Marquette University Honor Code.

\begin{table}[]
\centering
\begin{tabular}{|c|c|c|p{9cm}|c|}
\hline
Class \# & Week & Date & Topics & Chapters/Sections \\
\hline
1 & 1 & Tu Aug 26 & Introduction, Syllabus, Math Review, Statistics, Definitions & Math Review, 1 \\
2 & & Th Aug 28 & Graphs (pie diagram, bar graphs, stem-and-leaf displays, dot plot), Frequency distributions and histograms, Measures of Central Tendency, Measures of Dispersion & 2.1, 2.2, 2.3, 2.4 \\
\hline
3 & 2 & Tu Sep 2 & Measures of Position, Box-Plot, z-scores, Bivariate Data & 2.5, 3.1 \\
4 & & Th Sep 4 & Linear Correlation, Linear Regression & 3.2, 3.3 \\
\hline
5 & 3 & Tu Sep 9 & Probability of Events, Conditional Probability & 4.1, 4.2 \\
6 & & Th Sep 11 & Rules of Probability, Mutually Exclusive, Independent Events & 4.3, 4.4, 4.5 \\
\hline
7 & 4 & Tu Sep 16 & Random Variable, Discrete Random Variable, Binomial Probability Distribution & 5.1, 5.2, 5.3 \\
8 & & Th Sep 18 & Normal Distribution, Standard Normal Distribution & 6.1, 6.2, 6.3 \\
\hline
9 & 5 & Tu Sep 23 & Review Chapters 1-5 for Exam 1 & 2-5 \\
10 & & Th Sep 25 & \centering{\textbf{Exam 1}} & \\
\hline
11 & 6 & Tu Sep 30 & Return and Go Through Exam 1& \\
12 & & Th Oct 2 & Sampling Distributions, The Sampling Distribution of Sample Means, Application of the Sampling Distribution of Sample Means & 7.1, 7.2, 7.3 \\
\hline
13 & 7 & Tu Oct 7 & The Nature of Estimation, Estimation of Mean $\mu$ ($\sigma$ known) & 8.1, 8.2 \\
14 & & Th Oct 9 & Estimation of Mean $\mu$ ($\sigma$ known), Hypothesis Test of $\mu$ ($\sigma$ known): p-value approach & 8.3, 8.4 \\
\hline
15 & 8 & Tu Oct 14 & Hypothesis Test of $\mu$ ($\sigma$ known): classical approach & 8.5 \\
& & Th Oct 16 & \centering\textbf{Fall Break} & \\
\hline
16 & 9 & Tu Oct 21 & Review Chapters 6-8 for Exam 2 & 6-8 \\
17 & &Th Oct 23 & \centering\textbf{Exam 2} & \\
\hline
18 & 10 & Tu Oct 28 & Return and Go Through Exam 2 &  \\
19 & &Th Oct 30 & Inference about the mean $\mu$ ($\sigma$ unknown) & 9.1 \\
\hline
20 & 11 & Tu Nov 4 & Inference about the Binomial Probability of Success & 9.2 \\
21 & &Th Nov 6 & Inference about the Variance and Standard Deviation & 9.3 \\
\hline
22 & 12 & Tu Nov 11 & Dependent and Independent Samples, Inferences Concerning the Mean Difference Using Two Dependent Samples & 10.1, 10.2 \\
23 & &Th Nov 13 & Inferences concerning the Difference between Means Using Two Independent Samples & 10.3 \\
\hline
24 & 13 & Tu Nov 18 & Inferences Concerning the Difference between Proportions Using Two Independent Samples & 10.4, 10.5 \\
25 & & Th Nov 20 & Chi-Square Statistic, Inferences Concerning Multinomial Experiments & 11.1, 11.2, 11.3 \\
\hline
26 & 14 & Tu Nov 25 & Introduction to the ANOVA, Logic Behind ANOVA & 12.1, 12.2 \\
& &Th Nov 30 & \centering\textbf{Thanksgiving Break} & \\
\hline
27 & 15 & Tu Dec 25 & Application of Single-Factor ANOVA & 12.3 \\
28 & &Th Dec 4 & Review Chapters 9-12 for Final Exam & 9-12 \\
\hline
29 & &Fr Dec 12 & Final Exam 3:30 pm-5:30 pm & \\
\hline
\end{tabular}
\caption{\bf{\color{red}Tentative} Course Schedule}

\label{tab:schedule}
\end{table}

\end{description}
\end{document}
